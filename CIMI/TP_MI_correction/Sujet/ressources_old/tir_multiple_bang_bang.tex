
On consid\`ere le probl\`eme de contr\^ole optimal \eqref{eq:ocpBangReg}$_{|\veps=0}$. Dans ce cas, le contr\^ole optimal poss\`ede une discontinuit\'e
et nous allons donc utiliser une m\'ethode dite de tir multiple. On introduit le param\`etre \cmd{iarc} dans les fonctions \cmd{control}, \cmd{hvfun},
\cmd{exphvfun} et \cmd{expcost} qui correspond \`a l'indice de l'arc courant. On notera $y = (p_0, t_1, z_1)$ l'inconnue de la fonction de tir,
o\`u $p_0$ est le vecteur adjoint initial, $t_1$ est l'instant de transition entre l'arc 1 et l'arc 2 et o\`u $z_1$ est la valeur de l'\'etat et \'etat adjoint
\`a cet instant. La fonction de tir $S \colon \R^4 \to \R^4$ aura une \'equation permettant d'atteindre la cible $x_f$, une \'equation permettant la commutation
entre les arcs 1 et 2, et une derni\`ere \'equation (dite de raccordement) assurant la continuit\'e de l'\'etat et de l'\'etat adjoint \`a l'instant $t_1$.

\begin{myremark}
    Attention, les commandes \cmd{expcost} pour le calcul du co\^ut et \cmd{ssolve} sont d\'ej\`a \'ecrites !
\end{myremark}

\begin{myQuestion}
    \label{question:bang_bang_hamiltonien}
    Donner le pseudo-hamiltonien associ\'e au probl\`eme \eqref{eq:ocpBangReg}$_{|\veps=0}$ et d\'eterminer la fonction de commutation
    permettant la transition entre arcs $\arc_\pm$ et $\arc_0$.
\end{myQuestion}

\begin{myQuestion}
    \label{question:bang_bang_fonction_de_tir}
    D\'eterminer la fonction de tir $S$.
\end{myQuestion}

\begin{myExercice}
    \label{exo:tir_multiple} Tir multiple et contr\^ole bang-bang.
    \begin{enumerate}
        \item Se rendre dans le r\'epertoire \cmd{sujet4\_bang\_bang}.
        \item Impl\'ementer dans le r\'epertoire \cmd{lib} les fonctions \cmd{control}, \cmd{hvfun},
            \cmd{exphvfun}, \cmd{sfun}, \cmd{finiteDiff} et \cmd{sjac} (seulement la partie par diff\'erences finies).
            Attention, on a introduit l'argument \cmd{iarc} !
        \item Utiliser la BC-extr\'emale du probl\`eme \eqref{eq:ocpBangReg} avec $\veps$ petit (voir sujet \ref{chap:homotopie_discrete})
            pour initialiser la m\'ethode de tir (dans le fichier \cmd{main.m}) et r\'esoudre le syst\`eme d'\'equations $S(y) = 0$.
        \item Retrouver comme solution $p_0 \approx 0.270671$, $t_1 \approx 1.30685$ et $z_1 = (0,1)$ et v\'erifier que la valeur du crit\`ere est bien la valeur
            limite donn\'ee par la figure \ref{fig:results_tests_continuation_1}.
    \end{enumerate}
\end{myExercice}

\begin{myQuestion}
    \label{question:bang_bang_jacobienne}
    D\'eterminer la jacobienne de la fonction de tir.
\end{myQuestion}

\begin{myExercice}
    R\'esoudre le syst\`eme d'\'equations $S(y) = 0$ en utilisant l'IND de Bock pour le calcul de la jacobienne de la fonction de tir.
\end{myExercice}

