%%%%%%%%%%%%%%%%%%%%%%%%%%%%%%%%%%%%%%%%%%%%%%%%%%%%%%%%%%%%%%%%%%%%%%%%%%%%%%%%%%%%%%%%%%%%%%%%
%%%%%%%%%%%%%%%%%%%%%%%%%%%%%%%%%%%%%%%%%%%%%%%%%%%%%%%%%%%%%%%%%%%%%%%%%%%%%%%%%%%%%%%%%%%%%%%%
%\usepackage{pdfsync}
%\usepackage{hyperlatex}
\usepackage[utf8]{inputenc}
\usepackage{lmodern}
\usepackage[french]{babel}
\usepackage[T1]{fontenc}
\usepackage{verbatim}
\usepackage{xcolor}
\definecolor{linkcol}{rgb}{0.06,0.06,0.6}
\definecolor{citecol}{rgb}{0.6,0.04,0.04}
\definecolor{red}{rgb}{0.6,0.04,0.04}
\definecolor{blue}{rgb}{0.06,0.06,0.6}
\definecolor{grey}{rgb}{0.4,0.4,0.4}
\newcommand{\empha}[1]{\textbf{\textcolor{blue}{#1}}}
\newcommand{\emphb}[1]{\textcolor{red}{#1}}
\newcommand{\colorSection}{blue}
\newcommand{\colorSubSection}{blue}
\newcommand{\colorSubSubSection}{blue}
\newcommand{\colorChapter}{blue}

%Les chemins o\`u se trouvent les figures et les extensions de figures
% My pdf code
\usepackage{ifpdf}
\ifpdf
  \usepackage[pdftex]{graphicx}
  \DeclareGraphicsExtensions{.jpg,.pdf,.png}
  \usepackage[pagebackref,hyperindex=true,linktoc=all]{hyperref}
\else
  \usepackage{graphicx}
  \DeclareGraphicsExtensions{.ps,.eps}
  \usepackage[dvipdfm,pagebackref,hyperindex=true,linktocpage]{hyperref}
\fi
\hypersetup
{
pdfauthor="Olivier COTS", %auteur du document
pdfstartview={XYZ null null 1},
pdfhighlight=/O, %effet d'un clic sur un lien hypertexte
colorlinks=true, %couleurs sur les liens hypertextes
linkcolor=linkcol, %couleur des liens hypertextes internes
citecolor=citecol, %couleur des liens pour les citations
urlcolor=linkcol %couleur des liens pour les url
}
%\usepackage{graphicx}
\graphicspath{{\repRessource/figures/}}

%Les marges
\usepackage[a4paper,includeheadfoot,margin=2.54cm]{geometry}

%Hampath website
\newcommand{\homepage}{http://cots.perso.enseeiht.fr}
\newcommand{\resources}{\homepage/resources}

%Les maths et ams
\usepackage{amsfonts}
\usepackage{amscd}
\usepackage{amsthm}
\usepackage{amsmath}
\usepackage{amssymb}
\usepackage{mathrsfs}
\usepackage{mathtools}
\usepackage{xstring}
\usepackage{dsfont}

%Theorem style
%%-------------------
%\theoremstyle{plain}
%%-------------------
%\newtheorem{thrm}{Theorem}[section]
%\newtheorem{lmm}[thrm]{Lemma}
%\newtheorem{crllr}[thrm]{Corollary}
%\newtheorem{prpstn}[thrm]{Proposition}
%\newtheorem{crtrn}[thrm]{Criterion}
%\newtheorem{lgrthm}[thrm]{Algorithm}
%%------------------------
%\theoremstyle{definition}
%%------------------------
%\newtheorem{dfntn}[thrm]{Definition}
%\newtheorem{cnjctr}[thrm]{Conjecture}
%\newtheorem{xmpl}[thrm]{Example}
%\newtheorem{prblm}[thrm]{Problem}
%\newtheorem{rmrk}[thrm]{Remark}
%\newtheorem{nt}[thrm]{Note}
%\newtheorem{clm}[thrm]{Claim}
%\newtheorem{smmr}[thrm]{Summary}
%\newtheorem{cs}[thrm]{Case}
%\newtheorem{bsrvtn}[thrm]{Observation}
%
% Definitions des environnements des theoremes,...
%-------------------
\newtheoremstyle{mystyleTheorem}%       % Name
  {}%                                     % Space above
  {}%                                     % Space below
  {\itshape}%                             % Body font
  {}%                                     % Indent amount
  {}%                            % Theorem head font
  {.}%                                    % Punctuation after theorem head
  { }%                                    % Space after theorem head, ' ', or \newline
%  {}%                                     % Theorem head spec (can be left empty, meaning `normal')
  {\noindent\textbf{\thmname{#1}\thmnumber{~#2}}\thmnote{\color{blue}~(#3)}}%

\newtheoremstyle{mystyleDefinition}%       % Name
  {}%                                     % Space above
  {}%                                     % Space below
  {}%                             % Body font
  {}%                                     % Indent amount
  {}%                            % Theorem head font
  {.}%                                    % Punctuation after theorem head
  { }%                                    % Space after theorem head, ' ', or \newline
%  {}%                                     % Theorem head spec (can be left empty, meaning `normal')
  {\noindent\textbf{\thmname{#1}\thmnumber{~#2}}\thmnote{\color{blue}~(#3)}}%

\theoremstyle{mystyleTheorem}
%-------------------
\newtheorem{theorem}{Th\'eor\`eme}[chapter]         % Theoreme
\newtheorem{proposition}[theorem]{Proposition}      % Proposition
\newtheorem{lemma}[theorem]{Lemme}                  % Lemme
\newtheorem{cor}[theorem]{Corollaire}               % Corollaire d'un theoreme
%---------------------------------------------------------
%\newtheorem{question}{Question}                     % Question
\newtheorem{difficulty}{Difficult\'e}               % Difficult\'e
%------------------------

\theoremstyle{mystyleDefinition}
%------------------------
\newtheorem{definition}{D\'efinition}[chapter]      % Definition
%
\newtheorem{example}{Exemple}[chapter]              % Exemple
\newtheorem{counterexample}[example]{Contre-exemple}% Contre Exemple
\newtheorem{exercice}[example]{Exercice}            % Exercice
%
\newtheorem{assumption}{Hypoth\`ese}[chapter]       % Hypoth\`ese
%------------------------
\theoremstyle{remark}
%------------------------
\newtheorem{remark}{Remarque}[chapter]              % Remarque

%%%% Environnement pour les exercices %%%%%%%%%%%%%%%%%%%%%%%%%%%%%%%%%%%%%%%
%% \theoremstyle{plain}
\theoremstyle{mystyleDefinition}
\newtheorem{Exercice}{{\makebox[0pt][r]{$\rhd$\hspace*{1ex}}Exercice}}[chapter]
\newtheorem{Question}{Question}%[Exercice]
\newtheorem{SousQuestion}{\hspace*{1em}}[Question]
\newtheorem{Exemple}{Exemple}
\newcounter{exercice}
\newcounter{question}[exercice]
\renewcommand{\theexercice}{\arabic{exercice}}
\renewcommand{\thequestion}{\theexercice.\arabic{question}}
\theoremstyle{definition}
\newtheorem{Guide}{{\makebox[0pt][r]{$\blacktriangleright$\hspace*{1ex}}Guide}}
\renewcommand{\theGuide}{\hspace*{-0.9ex}}

%Tag \`a gauche ou \`a droite
\makeatletter
\newcommand{\reqnomode}{\tagsleft@false\let\veqno\@@eqno}
\newcommand{\leqnomode}{\tagsleft@true\let\veqno\@@leqno}
\makeatother

%Exigences
\reversemarginpar
%\newcommand{\attB}{\textcolor{green}{\textbullet}}
%\newcommand{\attM}{\textcolor{green}{\textbullet}\textcolor{orange}{\textbullet}}
%\newcommand{\attH}{\textcolor{green}{\textbullet}\textcolor{orange}{\textbullet}\textcolor{red}{\textbullet}}
\newcommand{\anoter}{ {\Large \textcolor{orange}{\ding{45}}} }

%Elements differentiel
%\newcommand{\diff}{\mathop{}\mathopen{}\mathrm{d}}
\newcommand{\xDif}{{\rm D}}
\newcommand{\xdif}{\,{\rm d}}
\newcommand{\diff}{\xdif}
\newcommand{\Diff}{{\rm \partial}}

%Les ensembles
%\newcommand{\U}{\mbox{$\mathbf{U}$}\xspace}
%\newcommand{\X}{\mbox{$\mathbf{X}$}\xspace}
%\newcommand{\M}{\mbox{$\mathbf{M}$}\xspace}
\newcommand{\U}{U}
\newcommand{\X}{X}
%\newcommand{\M}{M}
\newcommand{\Lcal}{\mathcal{L}}
\newcommand{\Ucal}{\mathcal{U}}
\newcommand{\I}{\mathcal{I}}
\newcommand{\A}{\mathcal{A}}
\renewcommand{\O}{\mathcal{O}}
\newcommand{\B}{\mathcal{B}}
\newcommand{\E}{\mathcal{E}}
\newcommand{\F}{\mathcal{F}}
\newcommand{\D}{\mathcal{D}}
\renewcommand{\P}{\mathcal{P}}
\newcommand{\V}{\mathcal{V}}
\newcommand{\xLn}[1]{{\rm L}^#1}
\newcommand{\xCn}[1]{{\rm C}^#1}

%Les softs et librairies
\newcommand{\pcr}[1]{{\fontfamily{pcr}\selectfont #1}}
%\newcommand{\vrb}[1]{{\fontfamily{ttfamily}\selectfont #1}}
\newcommand{\vrb}[1]{\texttt{#1}}
\newcommand{\cmd}[1]{\texttt{#1}}
\newcommand{\lib}[1]{\textsc{#1}}
\newcommand{\hampath}{\texttt{HamPath}}
\newcommand{\cotcot}{\texttt{cotcot}}
\newcommand{\matlab}{\lib{Matlab}}
\newcommand{\octave}{\lib{Octave}}
\newcommand{\fortran}{\lib{Fortran}}
\newcommand{\python}{\lib{Python}}
\newcommand{\interface}{\lib{Interface}}

%Intervalle
\newcommand{\intervalle}[4]{\mathopen{#1}#2
                                \mathclose{}\mathpunct{},#3
                                \mathclose{#4}}
\newcommand{\intervalleff}[2]{\intervalle{[}{#1}{#2}{]}}
\newcommand{\intervalleof}[2]{\intervalle{]}{#1}{#2}{]}}
%\newcommand{\intervalleof}[2]{\intervalle{(}{#1}{#2}{]}}
\newcommand{\intervallefo}[2]{\intervalle{[}{#1}{#2}{[}}
%\newcommand{\intervallefo}[2]{\intervalle{[}{#1}{#2}{)}}
\newcommand{\intervalleoo}[2]{\intervalle{]}{#1}{#2}{[}}
%\newcommand{\intervalleoo}[2]{\intervalle{(}{#1}{#2}{)}}
\newcommand{\intervalleentier}[2]{\intervalle{\llbracket}{#1}{#2}{\rrbracket}}

%L'arc de cercle pour les champs de Jacobi
\usepackage{yhmath}

%Misc
\newcommand\ie{\textit{i.e. }}
\newcommand\cf{\textit{cf. }}
\newcommand{\myemph}[1]{\textit{#1}}
\newcommand{\refcite}[1]{r\'ef.~\cite{#1}}
\newcommand{\refscite}[1]{r\'efs.~\cite{#1}}
\newcommand{\figsref}[1]{figs.~\ref{#1}}
\newcommand{\figref}[1]{fig.~\ref{#1}}
\newcommand{\frp}[2]{\frac{\Diff #1}{\Diff #2}}
\newcommand{\frpp}[2]{{\Diff #1}/{\Diff #2}}

%La commande pour la todo_list
\newcommand{\todo}[1]{\noindent {\color{red} \it TODO!!! #1}~\\}

%Ensembles de nombres
\newcommand{\nbSet}[1]{\mathbb{#1}}
\newcommand{\setPositive}{\text{\bf{\tiny+}}}
\newcommand{\setNegative}{\mathbb{\tiny-}}
\newcommand{\setStar}{\text{*}}
\newcommand{\M}{\nbSet{M}}
\newcommand{\N}{\nbSet{N}}
\newcommand{\Z}{\nbSet{Z}}
\newcommand{\Q}{\nbSet{Q}}
\newcommand{\R}{\nbSet{R}} %\newcommand{\R}{\mathbb{R}}
\newcommand{\C}{\nbSet{C}}


\newcommand{\setDeco}[2]{
    \IfEqCase{#2}{
        {s}{\nbSet{#1}^{\setStar}}
        {n}{\nbSet{#1}^{\phantom{\setStar}}_{\setNegative}}
        {p}{\nbSet{#1}^{\phantom{\setStar}}_{\setPositive}}
        {sn}{\nbSet{#1}^{\setStar}_{\setNegative}}
        {sp}{\nbSet{#1}^{\setStar}_{\setPositive}}
    }
}

\newcommand{\Ns}{ \ensuremath{\setDeco{N}{s}} }
\newcommand{\Rn}{ \ensuremath{\setDeco{R}{n}} }
\newcommand{\Rp}{ \ensuremath{\setDeco{R}{p}} }
\newcommand{\Rs}{ \ensuremath{\setDeco{R}{s}} }
\newcommand{\Rsp}{ \ensuremath{\setDeco{R}{sp}} }
\newcommand{\Rsn}{ \ensuremath{\setDeco{R}{sn}} }

%Valeur absolue et norme
\newcommand{\abs}[1]{\lvert#1\rvert} 
\newcommand{\absStyle}[1]{\left\lvert#1\right\rvert}
\newcommand{\norme}[1]{\lVert#1\rVert}
\newcommand{\normeStyle}[1]{\left\lVert#1\right\rVert}

%Petit o et grand O
\newcommand{\petito}[1]{o\mathopen{}\left(#1\right)}
\newcommand{\grandO}[1]{O\mathopen{}\left(#1\right)}

%Ensemble des .. tels que ..
\newcommand{\enstq}[2]{\left\{#1\mathrel{}\middle|\mathrel{}#2\right\}}

%Produit scalaire
\newcommand{\prodscal}[2]{\left(#1 \,|\, #2\right)}
\newcommand{\crochetDualite}[2]{\left\langle#1,#2\right\rangle}

%Fl\`eche
\usepackage[f]{esvect}
%\newcommand\vvec{\vec}
\newcommand\vvec{\overrightarrow}
\renewcommand\tilde{\widetilde}

%Solutions du PMP
\newcommand\lsol{\bar{\lambda}}
\newcommand\xsol{\bar{x}}
\newcommand\ysol{\bar{y}}
\newcommand\usol{\bar{u}}
\newcommand\zsol{\bar{z}}
\newcommand\psol{\bar{p}}
\newcommand\tfsol{\bar{t}_f}
\newcommand\tsol{\bar{t}\hspace{0.1em}}

% Lettres avec tilde
\def\xt{\widetilde{x}}
\def\ut{\widetilde{u}}
\def\pt{\widetilde{p}}
\def\ft{\widetilde{f}}

%Op\'erateur math
\DeclareMathOperator{\argmax}{arg\,max}
%\DeclareMathOperator{\rank}{rank}
\DeclareMathOperator{\codim}{codim}
\DeclareMathOperator{\rang}{rang}
\DeclareMathOperator{\rank}{rang}
\DeclareMathOperator{\vect}{Vect}
\DeclareMathOperator{\Isom}{Isom}
\DeclareMathOperator{\im}{Im}
\DeclareMathOperator{\supp}{supp}
\DeclareMathOperator{\sign}{sign}
\DeclareMathOperator{\comp}{num}

%exponential mapping
\newcommand{\expmap}[3]{\exp({#2 #3}) (#1)}

%Fonction homotopique
\renewcommand{\hom}{\Phi}

%Epsilon, phi, theta, etc.
\newcommand{\veps}{\varepsilon}
\newcommand\vphi{\phi}
\newcommand\pth{p_\theta}
\renewcommand\th{\theta}

\newcommand{\arc}{\gamma}

%Les dessins
\usepackage{tikz}
\usepackage{pgfkeys}
\usepackage{circuitikz}
\usetikzlibrary{arrows,shapes,backgrounds,patterns}
\usepackage{pifont}
\usepackage{tikzgraphicx}
\tikzstyle{every picture}+=[remember picture]
\tikzstyle{na} = [baseline=-.5ex]
\makeatletter
\newcommand{\gettikzxy}[3]{%
  \tikz@scan@one@point\pgfutil@firstofone#1\relax
  \edef#2{\the\pgf@x}%
  \edef#3{\the\pgf@y}%
}
\makeatother

% Définition des nouvelles options xmin, xmax, ymin, ymax
% Valeurs par défaut : -3, 3, -3, 3
\tikzset{
    xmin/.store in=\xmin, xmin/.default=-3, xmin=-3,
    xmax/.store in=\xmax, xmax/.default=3, xmax=3,
    ymin/.store in=\ymin, ymin/.default=-3, ymin=-3,
    ymax/.store in=\ymax, ymax/.default=3, ymax=3,
}
% Commande qui trace la grille entre (xmin,ymin) et (xmax,ymax)
\newcommand {\grille}
    {\draw[help lines] (\xmin,\ymin) grid (\xmax,\ymax);}
% Commande \axes
\newcommand {\axes}[2] {
    \draw[->,gray] (\xmin,0) -- (\xmax,0);
    \draw[->,gray] (0,\ymin) -- (0,\ymax);
    \draw (\xmax,0) node[below]{#1};
    \draw (0,\ymax) node[left]{#2};
}
% Commande qui limite l’affichage à (xmin,ymin) et (xmax,ymax)
\newcommand {\fenetre}
    {\clip (\xmin,\ymin) rectangle (\xmax,\ymax);}

%Les tableaux
\usepackage{calc}
\usepackage{array}
\newcolumntype{C}[1]{>{\centering}p{#1}}
\newcolumntype{L}[1]{>{\raggedright}p{#1}}

%Tableau : \'epaisseur des lignes
\usepackage{booktabs}
\newcommand{\smallhrule}{\specialrule{0.02em}{0.25em}{0.25em}}
\newcommand{\medhrule}{\specialrule{0.05em}{0.3em}{0.3em}}
\newcommand{\bighrule}{\specialrule{0.1em}{0.3em}{0.5em}}

%Passage seulement pour la version prof
\newcommand{\noi}{\noindent}
\newcommand{\HRule}{\noi\rule{\linewidth}{0.3mm}}
\newcommand{\onlyForProf}[1]{
\ifVIEWALL
    {
    \newpage
    {\color{grey}
        \noindent\fbox{\parbox{\linewidth-2\fboxrule-2\fboxsep}{\centering \textbf{Remarques personnelles}}}
        ~\\
        #1
        ~\\
        \HRule
    }
    \newpage
    } \else
\fi}

%Pour avoir des notes dans la marge
\newcommand{\noteInMarge}[1]{\ifVIEWALL{ \marginpar{{\color{red} #1}} }\else\fi}


%-------------------------------------------------------------------------------------------------------------------------------
%-------------------------------------------------------------------------------------------------------------------------------
%Pour le style du manuscrit

\usepackage{fancyhdr}                    % Fancy Header and Footer
\usepackage{colortbl}

% Clear Header Style on the Last Empty Odd pages
\makeatletter
\def\cleardoublepage{
    \clearpage
    \if@twoside 
        \ifodd
            \c@page
        \else%
            \hbox{}%
            \thispagestyle{empty}%              % Empty header styles
            \newpage%
            \if@twocolumn\hbox{}\newpage
            \fi
        \fi
    \fi}
\makeatother

%minitoc
%\usepackage[nottoc, notlof, notlot]{tocbibind}
\usepackage[french]{minitoc}
\setcounter{minitocdepth}{2}
\setlength{\mtcindent}{18pt} 
\renewcommand{\mtcfont}{\small\rm} 
\renewcommand{\mtcSfont}{\small\bf}

%\let\minitocORIG\minitoc
%\renewcommand{\minitoc}{\minitocORIG \vspace{1.5em}}

%table des mati\`eres et compteurs de titres
\setcounter{secnumdepth}{4}
\setcounter{tocdepth}{2}

% centered page environment
\newenvironment{vcenterpage}
{\newpage\vspace*{\fill}\thispagestyle{empty}\renewcommand{\headrulewidth}{0pt}}
{\vspace*{\fill}}

%nicer backref links
\renewcommand*{\backref}[1]{}
\renewcommand*{\backrefalt}[4]{%
\ifcase #1 %
%
\or
$\hookleftarrow$~#2.%
\else
$\hookleftarrow$~#2.%
\fi}
\renewcommand*{\backrefsep}{, }
\renewcommand*{\backreftwosep}{ et~}
\renewcommand*{\backreflastsep}{ et~}
%-------------------------------------------------------------------------------------------------------------------------------
%-------------------------------------------------------------------------------------------------------------------------------

%------------------------------------------------------------------------------------------------------------%
%Compteur pour les probl\`emes
\usepackage{totcount}
\newcounter{problem}%[chapter]
\regtotcounter{problem}
\newcommand{\tagProblem}[1][]{%
    \addtocounter{problem}{1}
    \tag{P\theproblem#1}
    %\tag{P\thechapter.\theproblem#1}
}
%Compteur pour les hypoth\`eses
\newcounter{assumptionx}[chapter]
\regtotcounter{assumptionx}
\newcommand{\tagAssumption}{%
    H\thechapter.\theassumptionx
}

\newenvironment{myassumption}
{\addtocounter{assumptionx}{1}
 \renewcommand\theassumption{\tagAssumption}
 \addtocounter{assumption}{-1}
 \begin{flushleft}
 \begin{tabular}{|| L{0.94\textwidth}}
  \assumption
  \vspace{-1em}
}
{\vspace{-1em}
\endassumption
\end{tabular}
 \end{flushleft}
 }

% \newenvironment{myassumption}{
%    \begin{flushright}
%        \begin{tabular}{|| L{0.94\textwidth} }
%            %\begin{assumption}
%            \textbf{Hypoth\`ese \tagAssumption.}
%            %\vspace{-1em}
%    }
%    {
%            %\vspace{-1em}
%            %\end{assumption}
%        \end{tabular}
%    \end{flushright}
%}

%Remarque perso
%\newcommand{\myremark}[1]{
%    \begin{flushright}
%        \begin{tabular}{| L{0.94\textwidth} }
%            \begin{remark}
%            \vspace{-1em}
%                #1
%            \vspace{-1em}
%            \end{remark}
%        \end{tabular}
%    \end{flushright}
%}
\newenvironment{myremark}{
    \begin{flushright}
        \begin{tabular}{| L{0.94\textwidth} }
            \begin{remark}
            \vspace{-1em}
    }
    {
            \vspace{-1em}
            \end{remark}
        \end{tabular}
    \end{flushright}
}

\newenvironment{myQuestion}{
    \begin{flushright}
        \begin{tabular}{| L{0.94\textwidth} }
            \begin{Question}
            \vspace{-1em}
    }
    {
            \vspace{-1em}
            \end{Question}
        \end{tabular}
    \end{flushright}
}

\newenvironment{myExercice}{
    \begin{flushright}
        \begin{tabular}{| L{0.94\textwidth} }
            \begin{Exercice}
            \vspace{-1em}
    }
    {
            \vspace{-1em}
            \end{Exercice}
        \end{tabular}
    \end{flushright}
}

%%%%%%%%%%%%%%%%%%%%%%%%%%%%%%%%%%%%%%%%
\usepackage[many]{tcolorbox}% http://ctan.org/pkg/tcolorbox
%\definecolor{mycolor}{rgb}{0.2,0.4,0.6}% Rule colour
%\definecolor{mycolor}{rgb}{0.122, 0.435, 0.698}% Rule colour
\makeatletter
%\newcommand{\myfbox}[2]{%
%  \setbox0=\hbox{#2}%
%  %\setlength{\@tempdima}{\dimexpr\wd0+13pt}%
%  \setlength{\@tempdima}{#1\textwidth}%
%  \begin{tcolorbox}[colframe=mycolor,boxrule=0.5pt,arc=4pt,
%      left=6pt,right=6pt,top=6pt,bottom=6pt,boxsep=0pt,width=\@tempdima]
%    #2
%  \end{tcolorbox}
%}
\newcommand{\myfbox}[4][]{%
%  \setbox0=\hbox{#3}%
    \setlength{\@tempdima}{#2\textwidth}%
    %\begin{tcolorbox}[colback=#3!2!white,colframe=#3,boxrule=0.5pt,arc=4pt,left=6pt,right=6pt,
    \begin{tcolorbox}[colback=white,colframe=#3,boxrule=0.8pt,arc=6pt,left=6pt,right=6pt,
          top=6pt,bottom=6pt,boxsep=0pt,width=\@tempdima,#1]
        #4
    \end{tcolorbox}
}
\makeatother

%\newcommand{\envbox}[2][theorem]{%
%    \begin{center}
%    \begin{minipage}[h!]{0.98\textwidth}
%        \myfbox{1.0}{blue}{
%        \begin{#1}
%            #2
%        \end{#1}
%        }
%    \end{minipage}
%    \end{center}
%}

\usepackage{xargs}
\newcommandx{\envbox}[3][1=theorem,2=]{%
    \begin{center}
    \begin{minipage}[h!]{0.999\textwidth}
        \myfbox{1.0}{blue}{
        \begin{#1}[#2]
            #3
        \end{#1}
        }
    \end{minipage}
    \end{center}
}

%Pr\'erequis
\newcommand{\prerequis}[1]{
%    \noindent\fbox{\parbox{\linewidth-2\fboxrule-2\fboxsep}{\textbf{Pr\'e-requis :} #1}}
%    \vspace{0.5em}
    \begin{center}
    \begin{minipage}[h!]{0.98\textwidth}
        \myfbox{1.0}{gray}{
        \textbf{Pr\'e-requis :} #1
        }
    \end{minipage}
    \end{center}
}

%%%%%%%%%%%%%%%%%%%%%%%%%%%%%%%%%%%%%%%%

%Proof
\makeatletter
\renewenvironment{proof}[1][\proofname]{
    \par\pushQED{\qed}\small\normalfont\topsep6\p@\@plus6\p@\relax\trivlist\item[\hskip\labelsep\itshape#1\@addpunct{.}]\ignorespaces
}{
    \popQED\endtrivlist\@endpefalse
}
\makeatother

%Les itemize
\usepackage{marvosym}
\usepackage{enumitem}
\frenchbsetup{CompactItemize=false}
\setlist{itemsep=0.25em}
\setitemize[1]{label={\small \Football}}
\setitemize[2]{label=$\diamond$}

\newcommand*\circled[1]{\tikz[baseline=(char.base)]{
    \node[shape=circle,draw,inner sep=2pt] (char) {#1};}}

\newcommand{\pgftextcircled}[1]{
    \setbox0=\hbox{#1}%
    \dimen0\wd0%
    \divide\dimen0 by 2%
    \begin{tikzpicture}[baseline=(a.base)]%
        \useasboundingbox (-\the\dimen0,0pt) rectangle (\the\dimen0,1pt);
        \node[circle,draw,outer sep=0pt,inner sep=0.1ex] (a) {#1};
    \end{tikzpicture}
}

%Les enumerations
%\usepackage{enumerate}

\usepackage{listings}
\definecolor{mymauve}{rgb}{0.58,0,0.82}
\lstset{ %
  backgroundcolor=\color{white},   % choose the background color; you must add \usepackage{color} or \usepackage{xcolor}
  basicstyle=\small,               % the size of the fonts that are used for the code
  breakatwhitespace=false,         % sets if automatic breaks should only happen at whitespace
  breaklines=true,                 % sets automatic line breaking
  captionpos=b,                    % sets the caption-position to bottom
  commentstyle=\color{blue},       % comment style
  deletekeywords={...},            % if you want to delete keywords from the given language
  escapeinside={\%*}{*)},          % if you want to add LaTeX within your code
  extendedchars=true,              % lets you use non-ASCII characters; for 8-bits encodings only, does not work with UTF-8
  frame=single,                    % adds a frame around the code
  keepspaces=true,                 % keeps spaces in text, useful for keeping indentation of code (possibly needs columns=flexible)
  keywordstyle=\color{red},        % keyword style
  otherkeywords={*,...},           % if you want to add more keywords to the set
  numbers=none,                    % where to put the line-numbers; possible values are (none, left, right)
  numbersep=5pt,                   % how far the line-numbers are from the code
  numberstyle=\tiny\color{gray},   % the style that is used for the line-numbers
  rulecolor=\color{black},         % if not set, the frame-color may be changed on line-breaks within not-black text (e.g. comments (green here))
  showspaces=false,                % show spaces everywhere adding particular underscores; it overrides 'showstringspaces'
  showstringspaces=false,          % underline spaces within strings only
  showtabs=false,                  % show tabs within strings adding particular underscores
  stepnumber=2,                    % the step between two line-numbers. If it's 1, each line will be numbered
  stringstyle=\color{mymauve},     % string literal style
  tabsize=2,                       % sets default tabsize to 2 spaces
  %title=\lstname,                 % show the filename of files included with \lstinputlisting; also try caption instead of title
  frameround=tttt                  % frameround=⟨t|f⟩⟨t|f⟩⟨t|f⟩⟨t|f⟩ ffff
                                   % The four letters designate the top right, bottom right, bottom left and top left corner. 
                                   % In this order. t makes the according corner round.
}
