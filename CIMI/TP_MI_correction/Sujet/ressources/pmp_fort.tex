Dans ce chapitre nous pr\'esentons le \emph{principe du maximum de Pontryagin} :
\begin{quotation}
\newblock L.~S. Pontryagin, V.~G. Boltyanski{\u\i}, R.~V. Gamkrelidze \& E.~F. Mishchenko,
\newblock \emph{The Mathematical Theory of Optimal Processes},
\newblock Translated from the Russian by K. N. Trirogoff, edited by L. W. Neustadt, Interscience Publishers John Wiley \& Sons, Inc., New York-London, 1962.
\end{quotation}
Pour un probl\`eme de contr\^ole optimal donn\'e, ce principe nous fournit des conditions n\'ecessaires d'optimalit\'e
que doit v\'erifier toute solution de ce probl\`eme.

%-----------------------------------------------------------------------------------------------------------------------------------------------
%-----------------------------------------------------------------------------------------------------------------------------------------------
\section{Formulation du probl\`eme de contr\^ole optimal}

Consid\'erons le probl\`eme de contr\^ole optimal suivant~:
\leqnomode
\begin{equation}
\tag{$\mathrm{OCP}$}
    \left\{ 
        \begin{array}{l}
            \displaystyle J(x_0,u(\cdot),t_0,t_f)   \coloneqq   \displaystyle g(t_0, x_0, t_f, x(t_f)) +
                                                \int_{t_0}^{t_f} f^0(t,x(t),u(t)) \, \diff t\longrightarrow \min \\[1.0em]
            \dot{x}(t)                      =   \displaystyle f(t,x(t),u(t)), 
                                                \quad  u(t) \in \U, 
                                                \quad t \in \intervalleff{t_0}{t_f} \text{ p.p.},
                                                \quad x(t_0) = x_0, \\[1.0em]
            c(t_0,x_0,t_f,x(t_f)) = 0,
        \end{array}
    \right. 
\label{eq:ocpGeneral}
\end{equation}
\reqnomode

C'est un probl\`eme tr\`es g\'en\'eral sous la forme de Bolza avec une dynamique non autonome, 
dont les instants initial et final sont libres, o\`u l'on consid\`ere des conditions aux limites m\'elang\'ees
et o\`u le contr\^ole peut \'eventuellement \^etre contraint. On d\'efinit les hypoth\`eses suivantes.

\begin{myassumption}
    \label{hyp:pmp}
    Consid\'erons une EDO contr\^ol\'ee non autonome $\dot{x}(t) = f(t,x(t),u(t)),$ o\`u $f$ est une application de classe
    $\xCn{1}$ de $\I \times \Omega \times \U$ dans $\R^n$, $\I$ un intervalle ouvert de $\R$, $\Omega$ un ouvert de $\R^n$
    et $\U$ un ensemble \empha{quelconque} de $\R^m$. Consid\'erons de plus une fonction $f^0$ de classe
    $\xCn{1}$ sur $\I \times \Omega \times \U$ et deux applications\footnotemark\ $g$ et $c$ de classe $\xCn{1}$ de
    $\I \times \Omega \times \I \times \Omega$ respectivement sur $\R$ et $\R^p$, $p \le 2(n+1)$.
    Notons $$X_c \coloneqq \enstq{(t_0,x_0,t_f,x_f) \in \I \times \Omega \times \I \times \Omega}{c(t_0,x_0,t_f,x_f) = 0}$$
    et supposons que $c'(t_0,x_0,t_f,x_f)$ est surjective pour tout
    $(t_0,x_0,t_f,x_f) \in X_c$.
\end{myassumption}
\footnotetext{On notera $(t_0, x_0, t_f, x_f)$ l'argument des applications $g$ et $c$. On \'ecrira donc $\frp{g}{x_f}$ la d\'eriv\'ee
        partielle de $g$ par rapport \`a la quatri\`eme variable.}

\begin{myremark}
    On peut supposer seulement que $f$ et $f^0$ sont $\xCn{0}$ par rapport \`a $t$ et $u$.
\end{myremark}

%\pagebreak

%
On cherche alors une solution $(x_0,u(\cdot),t_0,t_f)$, $x_0 \in \Omega$, $u(\cdot) \in L^\infty(\intervalleff{t_0}{t_f},\U)$, 
$0 \le t_0 \le t_f$ dans $\R$, telle que la trajectoire associ\'ee \empha{$\mathbf{t \mapsto x(t) \coloneqq x(t,t_0,x_0,u(\cdot))}$} soit d\'efinie sur 
$\intervalleff{t_0}{t_f}$ et telle que la contrainte diff\'erentielle, la contrainte sur le contr\^ole et les conditions
aux limites soient v\'erifi\'ees, et qui de plus minimise le crit\`ere.


%-------------------------------------------------------------------------------------------------------------------------
%-------------------------------------------------------------------------------------------------------------------------
\section{Principe du Maximum de Pontryagin}

\'Enon\c cons maintenant le Principe du Maximum de Pontryagin (PMP).

\envbox[theorem][Principe du Maximum de Pontryagin]{
    Si $(x_0,u(\cdot),t_0,t_f)$, avec $x(\cdot)$ la trajectoire associ\'ee, est solution du probl\`eme \eqref{eq:ocpGeneral} 
    sous les hypoth\`eses \ref{hyp:pmp},
    alors il existe un \emph{vecteur adjoint} $p(\cdot) \in AC(\intervalleff{t_0}{t_f},(\R^n)^*)$, un r\'eel $p^0\le 0$,
    tels que \emphb{$(p(\cdot),p^0)\ne(0,0)$}, et un multiplicateur \emphb{$\lambda \in (\R^p)^*$}, 
    tels que les \'equations suivantes sont v\'erifi\'ees pour $t\in\intervalleff{t_0}{t_f}$ p.p.~:
    \begin{equation}
        \label{eq:pmp_equations_hamilton}
        \begin{aligned}
            \dot{x}(t)              &= \phantom{-}\frp{H}{p}(t,x(t),p(t),p^0,u(t)), \\
            \dot{p}(t)              &=           -\frp{H}{x}(t,x(t),p(t),p^0,u(t)),
        \end{aligned}
    \end{equation}
    o\`u $
            H(t,x,p,p^0,u) \coloneqq p\, f(t,x,u) + p^0 \, f^0(t,x,u)
        $
    est le pseudo-hamiltonien associ\'e au probl\`eme \eqref{eq:ocpGeneral},
    et on a la condition de maximisation du hamiltonien suivante pour $t\in\intervalleff{t_0}{t_f}$ p.p.~:
    \begin{equation}
        \label{eq:pmp_condition_maximisation}
            H(t,x(t),p(t),p^0,u(t)) = \max_{\footnotesize w\in\U} H(t,x(t),p(t),p^0,w).
    \end{equation}
    Les conditions aux limites $c(t_0,x_0,t_f,x(t_f)) = 0$ et $x(t_0)=x_0$ sont v\'erifi\'ees et on a en plus les conditions de transversalit\'e suivantes~:
    \begin{equation}
        p(t_0) = - \left( \lambda \, \frp{c}{x_0} + p^0 \, \frp{g}{x_0} \right), \quad
        p(t_f) = \left( \lambda \, \frp{c}{x_f} + p^0 \, \frp{g}{x_f} \right),
        \label{eq:pmp_conditions_transversalite}
    \end{equation}
    appliqu\'e en $(t_0,x_0,t_f,x(t_f))$.
    %
    Enfin, puisque les temps initial et final sont libres, si $u(\cdot)$ est continu aux temps $t_0$, respectivement $t_f$, alors on a
    les conditions sur le hamiltonien suivantes~:
    \begin{equation}
        H[t_0] = \left( \lambda \, \frp{c}{t_0} + p^0 \, \frp{g}{t_0} \right), \quad
        H[t_f] = - \left( \lambda \, \frp{c}{t_f} + p^0 \, \frp{g}{t_f} \right),
%        H(x(t_f),p(t_f),p^0,u(t_f)) = - \left( \sum_{i=1}^{p} \mu_i \, \frp{c_i}{t}(t_f,x(t_f)) + p^0 \, \frac{\partial g}{\partial t}(t_f,x(t_f)) \right).
        \label{eq:pmp_conditions_hamiltonien}
    \end{equation}
    toujours appliqu\'e en $(t_0,x_0,t_f,x(t_f))$ et o\`u $[t] \coloneqq (t,x(t),p(t),p^0,u(t))$.
    \label{thm:pmp_fort_general}
}

\begin{myremark}
    On rappelle~:
    $\displaystyle \lambda \frp{c}{x_0} = \sum_{i=1}^{p} \lambda_i \, \frp{c_i}{x_0},$ avec $\lambda \coloneqq (\lambda_1, \cdots, \lambda_p)$.
\end{myremark}

\begin{myremark}
    La convention $p^0\le 0$ conduit au principe du \emph{maximum} tandis que $p^0 \ge 0$ conduit au principe du \emph{minimum}, \ie 
    la condition \eqref{eq:pmp_condition_maximisation} serait une minimisation.
\end{myremark}

\begin{myremark}
    Dans le cas o\`u $\U$ est un ouvert de $\R^m$, \ie lorsqu'il n'y a pas de contraintes sur le contr\^ole, 
    la condition de maximisation \eqref{eq:pmp_condition_maximisation} implique $\partial_u H[t] = 0$.%, car pour presque tout $t$,
    %le contr\^ole optimal $u(t)$ est un point critique de l'application partielle $u \mapsto H(t,x(t),p(t),p^0,u)$.
\end{myremark}

%-------------------------------------------------------------------------------------------------------------------------
%-------------------------------------------------------------------------------------------------------------------------
\section{D\'efinitions et propri\'et\'es importantes}

\envbox[definition]{ On introduit les d\'efinitions suivantes.
    \label{def:extremale_autres}
    \begin{itemize}
        \item Une \emph{extr\'emale} du probl\`eme \eqref{eq:ocpGeneral} est un quadruplet $(x(\cdot),p(\cdot),p^0,u(\cdot))$ 
            solution des \emph{\'equations hamiltoniennes} \eqref{eq:pmp_equations_hamilton} et de la \emph{condition de maximisation} 
            \eqref{eq:pmp_condition_maximisation}.
        \item On parle de \emph{BC--extr\'emale} (BC vient de ``Boundary Conditions'') si l'extr\'emale v\'erifie en plus
            les \emph{conditions aux limites} $c(t_0,x_0,t_f,x(t_f)) = 0$, $x(t_0) = x_0$,
            les \emph{conditions de transversalit\'e} \eqref{eq:pmp_conditions_transversalite}
            et les \emph{conditions sur le hamiltonien} \eqref{eq:pmp_conditions_hamiltonien}.
        \item Une extr\'emale $(x(\cdot),p(\cdot),p^0,u(\cdot))$ est dite \emph{anormale} si $p^0=0$ et \emph{normale} dans le cas contraire. 
            Dans le cas normal on peut fixer $p^0$ \`a $-1$ par exemple, par homog\'en\'eit\'e.
%            Une fois la valeur de $p^0$ fix\'e, on peut \'ecrire $H(t,x,p,u)$.
        \item Une extr\'emale d\'efinie sur un intervalle $I \subset \intervalleff{t_0}{t_f}$, $t_0 < t_f$, est dite
            \emph{singuli\`ere} si $\partial_{u} H[t]=0$ pour tout $t \in I$, elle est dite \emph{r\'eguli\`ere} sinon.
            On appelle \emph{arc bang}, une portion d'extr\'emale r\'eguli\`ere sur laquelle le contr\^ole $u(t)$ est de norme constante.
        %\item Si le hamiltonien est nul le long d'une extr\'emale alors elle est appel\'ee \emph{exceptionnelle}.
    \end{itemize}
}

\begin{myremark}
    Le probl\`eme important du \emph{temps minimal} correspond \`a 
    $
        f^0 \equiv 1$ et $g \equiv 0
    $
    ou bien \`a
    $
        f^0 \equiv 0$ et $g(t_0,x_0,t_f,x_f) = t_f.
    $
    Dans tous les cas, les conditions de transversalit\'e obtenues sont bien les m\^emes.
\end{myremark}

On peut de plus d\'efinir un v\'eritable hamiltonien et son syst\`eme hamiltonien 
sous certaines conditions, d'apr\`es la proposition suivante.
\envbox[proposition]{
    \label{prop:ham_vrai}
    Soit $(\xsol(\cdot),\psol(\cdot),p^0,\usol(\cdot))$ une extr\'emale du probl\`eme \eqref{eq:ocpGeneral}.
    On note $z(\cdot) \coloneqq (x(\cdot),p(\cdot))$. Si pour presque tout $t \in \intervalleff{t_0}{t_f}$, le \empha{hamiltonien maximis\'e}
    \[
        z\coloneqq(x,p) \mapsto h(t,z) \coloneqq \max_{\footnotesize u\in\U} H(t,x,p,p^0,u)
    \]
    est d\'efini et lisse sur un voisinage de l'extr\'emale, 
    alors pour presque tout $t \in \intervalleff{t_0}{t_f}$
    \begin{equation*}
        \dot \zsol(t) = \vvec{h}(t,\zsol(t)) \coloneqq \left( \frp{h}{p}(t,\zsol(t)),-\frp{h}{x}(t,\zsol(t)) \right),
    \end{equation*}
    et $h(t,z)$ d\'efinit un v\'eritable hamiltonien (il ne d\'epend pas de $u$).
}



