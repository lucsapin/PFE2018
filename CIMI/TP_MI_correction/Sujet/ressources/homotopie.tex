Consid\'erons le probl\`eme de contr\^ole optimal suivant (double int\'egrateur, minimisation $L^1$) :
\leqnomode
\begin{equation*}
\tagProblem
    \left\{ 
        \begin{array}{l}
            \displaystyle J(u(\cdot)) \coloneqq \displaystyle \int_0^{t_f} \abs{u(t)} \, \diff t\longrightarrow \min            \\[1.0em]
            \dot{x}_1(t)    =  \displaystyle x_2(t)                                                                                     \\[0.5em]
            \dot{x}_2(t)    =  \displaystyle u(t), \quad \abs{u(t)} \le u_\mathrm{max}, \quad t \in \intervalleff{0}{t_f} \text{ p.p.}, \\[1.0em]
            x(0) = x_0 , \quad x(t_f) = x_f,
        \end{array}
    \right. 
    \label{eq:ocp_DI_min_L1}
\end{equation*}
\reqnomode
avec $t_f \coloneqq 3$, $x_0 \coloneqq (-1,0)$, $x_f \coloneqq (0,0)$ et $\forall\, t \in\intervalleff{0}{t_f}$, $x(t) = (x_1(t), x_2(t)) \in \R^2$.
%
Notons 
\[
    H(x,p,p^0,u) \coloneqq p_1\, x_2 + p_2\, u + p^0\, \abs{u}, \quad x = (x_1, x_2), \quad p = (p_1, p_2),
\]
le pseudo-hamiltonien. On consid\`ere le cas normal et on fixe $p^0 = -1$. La condition de maximisation
nous donne comme loi de contr\^ole~:
\begin{equation*}
    u(t) = 
    \left\{
        \begin{array}{ll}
            \sign(p_2(t))\, u_\mathrm{max}      & \text{ si } \abs{p_2(t)}+p^0 > 0, \\[0.5em]
            0                   & \text{ si } \abs{p_2(t)}+p^0 < 0, \\[0.5em]
            ?                   & \text{ si } \abs{p_2(t)}+p^0 = 0.
        \end{array}
    \right.
\end{equation*}
%

Nous voulons ici r\'esoudre le probl\`eme par une m\'ethode de tir multiple. Nous avons donc besoin dans un premier temps, 
de d\'eterminer la structure du contr\^ole. Pour ce faire, nous pourrions utiliser un m\'ethode de tir simple en prenant certaines
pr\'ecautions, notamment sur le calcul de la jacobienne de la fonction de tir. Une autre possibilit\'e est d'utiliser une
m\'ethode homotopique diff\'erentielle de suivi de chemin combin\'ee \`a une m\'ethode de r\'egularisation.
%
Nous introduisons donc le crit\`ere suivant afin de r\'egulariser le contr\^ole optimal : 
\begin{equation}
    J_\lambda(u(\cdot)) \coloneqq \displaystyle \int_0^{t_f} \abs{u(t)}^{2-\lambda} \, \diff t, \quad \lambda \in \intervallefo{0}{1}.
    \label{eq:critere_regularisant}
\end{equation}

%---------------------------------------------------------------------------------------------------------
%---------------------------------------------------------------------------------------------------------
\section{R\'esolution du probl\`eme r\'egularis\'e}

Nous consid\'erons ici le probl\`eme \eqref{eq:ocp_DI_min_L1} o\`u l'on remplace le crit\`ere $J$ par $J_\lambda$ d\'efinit
en \eqref{eq:critere_regularisant}. Le pseudo-hamiltonien est donn\'e par :
\[
    H_\lambda(x,p,p^0,u) \coloneqq p_1\, x_2 + p_2\, u + p^0\, \abs{u}^{2-\lambda}
\]
et le contr\^ole maximisant s'\'ecrit
\[
    u(t) = \sign(p_2(t))\, \left( - \frac{\abs{p_2(t)}}{p^0\, (2-\lambda)} \right)^{\frac{1}{1-\lambda}}.
\]
La fonction de tir est simplement donn\'ee par :
\begin{equation*}
    \begin{array}{rlll}
        S \colon    & \R^2    & \longrightarrow   & \R^2 \\
                    & y       & \longmapsto       & S(y) \coloneqq x(t_f,x_0,y) - x_f.
    \end{array}
\end{equation*}

\begin{myExercice} Se rendre dans le r\'epertoire \cmd{TP3\_homotopie\_regularisation\_double\_integrateur/}\cmd{exo\_1\_regularisation}.
    \begin{enumerate}
        \item Compl\'eter seulement \cmd{hfun} dans \cmd{hfun.f90}.
        \item Compiler le probl\`eme.
        \item Jeter un \oe il au script \matlab\ \cmd{main31.m} puis l'ex\'ecuter.
    \end{enumerate}
\end{myExercice}

%---------------------------------------------------------------------------------------------------------
%---------------------------------------------------------------------------------------------------------
\section{R\'esolution du probl\`eme de minimisation $L^1$}

D'apr\`es la r\'egularisation (pour $\lambda$ proche de 1), le contr\^ole optimal solution du probl\`eme \eqref{eq:ocp_DI_min_L1}
poss\`ede trois arcs : $u=u_\mathrm{max}$ suivi de $u=0$ puis $u=-u_\mathrm{max}$.
On note $H_+(z) \coloneqq H(z, p^0, u_\mathrm{max})$, $H_-(z) \coloneqq H(z, p^0, -u_\mathrm{max})$, $H_0(z) \coloneqq H(z, p^0, 0)$
et $\vv{H_+}$, $\vv{H_-}$, $\vv{H_0}$ les syst\`emes hamiltoniens associ\'es.
La fonction de tir correspondant \`a cette structure s'\'ecrit :
\begin{equation*}
    \begin{array}{rlll}
        S \colon    & \R^4          & \longrightarrow   & \R^4 \\
        & y\coloneqq(p_0, t_1, t_2) & \longmapsto       & S(y) \coloneqq
        \begin{bmatrix}
            \Pi_{p_2}\left( z_1 \right) + p^0   \\[0.5em]
            \Pi_{p_2}\left( z_2 \right) - p^0   \\[0.5em]
            \Pi_x\left( z_f \right) - x_f
        \end{bmatrix},
    \end{array}
\end{equation*}
o\`u
\[
    z_1 \coloneqq \expmap{x_0,p_0}{t_1}{\vv{H_+}}, \quad z_2 \coloneqq \expmap{z_1}{(t_2-t_1)}{\vv{H_0}},
    \quad z_f \coloneqq \expmap{z_2}{(t_f-t_2)}{\vv{H_-}}.
\]

\begin{myExercice} Se rendre dans le r\'epertoire \cmd{TP3\_homotopie\_regularisation\_double\_integrateur/}\cmd{exo\_2\_min\_L1}.
    \begin{enumerate}
        \item Compl\'eter \cmd{control}, \cmd{hfun} et \cmd{sfun}, respectivement dans les fichiers \cmd{afun.f90}, \cmd{hfun.f90} et \cmd{sfun.f90}.
        \item Compiler le probl\`eme.
        \item Compl\'eter le script \matlab\ \cmd{main32.m} puis l'ex\'ecuter. Il faut fournir au solveur des \'equations de tir, la pr\'ediction
            initiale du vecteur adjoint initial $p_0$ et des instants de commutations $t_1$ et $t_2$. Utiliser pour cela, la solution du probl\`eme 
            r\'egularis\'e avec $\lambda$ proche de $1$.
    \end{enumerate}
\end{myExercice}
