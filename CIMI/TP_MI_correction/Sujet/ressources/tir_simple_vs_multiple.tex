%---------------------------------------------------------------------------------------------------------
%---------------------------------------------------------------------------------------------------------
\section{Double int\'egrateur : \'energie min}

Consid\'erons le probl\`eme de contr\^ole optimal suivant :
\leqnomode
\begin{equation*}
\tagProblem
    \left\{ 
        \begin{array}{l}
            \displaystyle J(u(\cdot)) \coloneqq \displaystyle \frac{1}{2} \int_0^{t_f} u(t)^2 \, \diff t\longrightarrow \min            \\[1.0em]
            \dot{x}_1(t)    =  \displaystyle x_2(t)                                                                                     \\[0.5em]
            \dot{x}_2(t)    =  \displaystyle u(t), \quad \abs{u(t)} \le u_\mathrm{max}, \quad t \in \intervalleff{0}{t_f} \text{ p.p.}, \\[1.0em]
            x(0) = x_0 , \quad x(t_f) = x_f,
        \end{array}
    \right. 
    \label{eq:ocp_DI_min_E}
\end{equation*}
\reqnomode
avec $t_f \coloneqq 1$, $x_0 \coloneqq (-1,0)$, $x_f \coloneqq (0,0)$ et $\forall\, t \in\intervalleff{0}{t_f}$, $x(t) = (x_1(t), x_2(t)) \in \R^2$.
Notons 
\[
    H(x,p,p^0,u) \coloneqq p_1\, x_2 + p_2\, u + p^0\, \frac{1}{2} u^2, \quad x = (x_1, x_2), \quad p = (p_1, p_2),
\]
le pseudo-hamiltonien. On consid\`ere le cas normal et on fixe $p^0 = -1$. La condition de maximisation
nous donne comme loi de contr\^ole~:
\begin{equation*}
    u(t) = 
    \left\{
        \begin{array}{ll}
            u_s(x(t),p(t)) \coloneqq p_2(t)   & \text{ si } \abs{p_2(t)} \le u_\mathrm{max}, \\[0.5em]
            +u_\mathrm{max}                      & \text{ si } p_2(t) > u_\mathrm{max}, \\[0.5em]
            -u_\mathrm{max}                      & \text{ si } p_2(t) < -u_\mathrm{max}.
        \end{array}
    \right.
\end{equation*}
%
La fonction de tir est donn\'ee par :
\begin{equation*}
    \begin{array}{rlll}
        S \colon    & \R^2    & \longrightarrow   & \R^2 \\
                    & y       & \longmapsto       & S(y) \coloneqq x(t_f,x_0,y) - x_f,
    \end{array}
\end{equation*}
o\`u $y$ joue ici le r\^ole du vecteur adjoint initial $p(0)$.

\begin{myExercice} Se rendre dans le r\'epertoire \cmd{TP\_2\_tir\_simple\_et\_multiple\_double\_integrateur/exo\_1\_energie\_min}.
    \begin{enumerate}
        \item Compl\'eter \cmd{control}, \cmd{hfun} et \cmd{sfun}, respectivement dans les fichiers \cmd{afun.f90}, \cmd{hfun.f90} et \cmd{sfun.f90}.
            Attention pour \cmd{sfun}, m\^eme si $x_f = (0,0)$, il est pass\'e en param\`etre \`a l'aide du vecteur \cmd{par} : $x_f = par(5:6)$.
        \item Compiler le probl\`eme.
        \item Jeter un \oe il au script \matlab\ \cmd{main21.m} puis l'ex\'ecuter. Vous pouvez tester la r\'esolution du probl\`eme pour deux valeurs de
            $u_\mathrm{max}$.
    \end{enumerate}
\end{myExercice}

\begin{myremark}
    \anoter
    La loi de commande $\usol(\cdot)$ associ\'ee \`a la solution des \'equations de tir est lisse lorsque $u_\mathrm{max} \ge 6$.
    Pour $u_\mathrm{max} < 6$, la loi de commande perd en r\'egularit\'e. Pour $u_\mathrm{max} = 4.5$, elle est continue mais pas $C^1$,
    seulement $C^1$ par morceaux (elle est m\^eme lisse par morceaux).
\end{myremark}

%---------------------------------------------------------------------------------------------------------
%---------------------------------------------------------------------------------------------------------
\section{Double int\'egrateur : temps min}

Consid\'erons le probl\`eme \eqref{eq:ocp_DI_min_E} o\`u l'on remplace le crit\`ere par le suivant : $J(u(\cdot), t_f) \coloneqq t_f$.
On remarque alors que dans ce cas, le temps final $t_f$ est libre. Fixons de plus la borne sur le contr\^ole : $u_\mathrm{max} = 1$.
%
Notons le pseudo-hamiltonien
\[
    H(x,p,u) \coloneqq p_1\, x_2 + p_2\, u.
\]
La condition de maximisation nous donne comme loi de contr\^ole~:
\begin{equation*}
    u(t) = 
    \left\{
        \begin{array}{ll}
            +u_\mathrm{max}  & \text{ si } p_2(t) > 0, \\[0.5em]
            -u_\mathrm{max}  & \text{ si } p_2(t) < 0, \\[0.5em]
            ?                & \text{ si } p_2(t) = 0.
        \end{array}
    \right.
\end{equation*}

\begin{myremark}
    \anoter
    Il n'est pas possible d'avoir $p_2(t) = 0$ sur un intervalle de temps non r\'eduit \`a un point. Ainsi, nous pouvons ignorer ce cas.
\end{myremark}

Puisque $t_f$ est libre, le PMP nous donne la condition de transversalit\'e \[H(x(t_f), p(t_f), u(t_f)) = -p^0, \]
o\`u l'on fixe toujours $p^0=-1$ car nous ne consid\'erons que le cas normal. 
%

%-----------------------------------------------------
\subsection{Tir simple}

La fonction de tir simple est donn\'ee par~:
\begin{equation*}
    \begin{array}{rlll}
        S \colon    & \R^3      & \longrightarrow   & \R^3 \\
        & y\coloneqq(p_0, t_f)  & \longmapsto       & S(y) \coloneqq
        \begin{bmatrix}
            x(t_f,x_0,p_0) - x_f \\
            H(x(t_f,x_0,p_0),p(t_f,x_0,p_0),u(t_f))+p^0
        \end{bmatrix}.
    \end{array}
\end{equation*}

\begin{myExercice} Se rendre dans le r\'epertoire \cmd{TP\_2\_tir\_simple\_et\_multiple\_double\_integrateur/} \cmd{exo\_2\_temps\_min\_tir\_simple}.
    \begin{enumerate}
        \item Compl\'eter \cmd{control}, \cmd{hfun} et \cmd{sfun}, respectivement dans les fichiers \cmd{afun.f90}, \cmd{hfun.f90} et \cmd{sfun.f90}.
            Attention pour \cmd{sfun}, m\^eme si $x_f = (0,0)$, il est pass\'e en param\`etre \`a l'aide du vecteur \cmd{par} : $x_f = par(3:4)$.
        \item Compiler le probl\`eme.
        \item Jeter un \oe il au script \matlab\ \cmd{main22.m} puis l'ex\'ecuter.
    \end{enumerate}
\end{myExercice}

\begin{myremark}
    \anoter
    Remarquer que le solveur \cmd{fsolve} de \matlab\ ne fait qu'une it\'eration mais ne progresse pas.
\end{myremark}

%-----------------------------------------------------
\subsection{Tir multiple}

Nous introduisons quelques notations pour clarifier l'expression de la fonction de tir multiple que l'on d\'efinit ci-apr\`es.
On note $H_+(z) \coloneqq H(z, u_\mathrm{max})$, $H_-(z) \coloneqq H(z, -u_\mathrm{max})$ et $\vv{H_+}$, $\vv{H_-}$ les syst\`emes hamiltoniens associ\'es.
Pour d\'efinir la fonction de tir multiple, nous devons n\'ecessairement conna\^itre la structure du contr\^ole. Ici, nous savons que la structure 
contient deux arcs : $+u_\mathrm{max}$ suivi de $-u_\mathrm{max}$. Le PMP nous donne qu'\`a l'instant de la commutation (not\'e $t_1$), on a $p_2(t_1) = 0$.
Nous pouvons maintenant d\'efinir la fonction de tir multiple (de structure), qui est donn\'ee par~:
\begin{equation*}
    \begin{array}{rlll}
        S \colon    & \R^4          & \longrightarrow   & \R^4 \\
        & y\coloneqq(p_0, t_f, t_1) & \longmapsto       & S(y) \coloneqq
        \begin{bmatrix}
            \Pi_x\left( z_f \right) - x_f \\[0.5em]
            H_-(z_f)+p^0                  \\[0.5em]
            \Pi_{p_2}\left( z_1 \right)
        \end{bmatrix},
    \end{array}
\end{equation*}
o\`u
\[
    z_1 \coloneqq \expmap{x_0,p_0}{t_1}{\vv{H_+}}, \quad z_f \coloneqq \expmap{z_1}{(t_f-t_1)}{\vv{H_-}}.
\]
On rappelle que $\Pi_x(z) = x$ avec $z=(x,p)$, $p=(p_1, p_2)$, et on d\'efinit $\Pi_{p_2}(z) = p_2$.

\begin{myExercice} Se rendre dans le r\'epertoire \cmd{TP\_2\_tir\_simple\_et\_multiple\_double\_integrateur/}
    \cmd{exo\_3\_temps\_min\_tir\_multiple\_de\_structure}.
    \begin{enumerate}
        \item Jeter un \oe il \`a la routine \cmd{control} de \cmd{afun.f90}. Noter l'utilisation de la variable \cmd{iarc} indiquant si l'on se trouve
            sur le premier o\`u le second arc.
        \item Jeter un \oe il \`a la routine \cmd{hfun} de \cmd{hfun.f90}. Elle reste inchang\'ee.
        \item Compl\'eter \cmd{sfun} de \cmd{sfun.f90}. Attention \`a bien utiliser la variable \cmd{iarc}.
        \item Compiler le probl\`eme.
        \item Jeter un \oe il au script \matlab\ \cmd{main23.m} puis l'ex\'ecuter.
    \end{enumerate}
\end{myExercice}

\begin{myremark}
    \anoter
    Comparer la pr\'ecision de la solution num\'erique entre le tir multiple et le tir simple. On notera que pour le tir multiple, nous avons
    besoin d'une information suppl\'ementaire : la structure du contr\^ole optimal.
\end{myremark}
