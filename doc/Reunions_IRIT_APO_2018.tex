\documentclass[fleqn,% draft,
a4paper,11pt]{scrbook}

% Document language : english

\newif\ifprinted
\printedfalse

% Police
\usepackage[utf8]{inputenc}

%% Serif font
% \IfFileExists{stix.sty}{%
%   \usepackage[upint]{stix}}%
% {\usepackage{lmodern}
%   \def\coloneq{:=}}

%% Sans font
%\usepackage[scaled=0.875]{helvet}

\usepackage[tt=false]{libertine}
\usepackage[T1]{fontenc}

\usepackage{amsthm}
\usepackage[slantedGreek,cmintegrals,bigdelims,vvarbb,libertine,libaltvw,liby]{newtxmath}
\useosf

\ifprinted
\KOMAoptions{BCOR=0.5cm}
\fi

\recalctypearea


\let\forall\forallAlt
\let\exists\existsAlt
\let\nexists\nexistsAlt

%% Monospace font
%\usepackage[scaled=0.92,varqu,varl]{zi4}
\usepackage[scaled=0.77]{beramono}


% \usepackage{libertine}
% \usepackage[varg,cmintegrals,bigdelims,varbb,libertine]{newtxmath}

\usepackage{ifdraft}

\usepackage{chngcntr}
\usepackage{tocvsec2}

\KOMAoptions{chapterprefix=true}
\KOMAoptions{headsepline=true}
\KOMAoptions{index=toc}
%\KOMAoptions{bibliography=totocnumbered}
%\addtokomafont{chapter}{\centering}
% \ifdraft{\KOMAoptions{twoside=false}}{}
\renewcommand*{\partformat}{\partname~\thepart}
\renewcommand*{\partpagestyle}{empty}
\renewcommand*{\chapterformat}{\chapappifchapterprefix~\thechapter\\\rule{\linewidth}{1.5pt}\\}


%\KOMAoptions{numbers=noenddot}
\renewcommand*{\figureformat}{\figurename~\thefigure}
\renewcommand*{\tableformat}{\tablename~\thetable}
%\renewcommand*{\captionformat}{.}

%\KOMAoptions{parskip=half*}

\newcommand{\chappreamble}[1]{\textit{\noindent #1}\leavevmode\vspace{\baselineskip}}

\newenvironment{chapabstract}{}{}



% \usepackage{titlesec}
% \titleformat{\chapter}[display]

\usepackage[final]{listings}

\usepackage[style=alphabetic,maxnames=4,backend=bibtex,
backref=true% ,noadjust
,
firstinits=true]{biblatex}
%   \makeatletter
%   \def\@citex#1[#2]#3{% 
%     \@safe@activestrue\edef\@tempd{#1}\@safe@activesfalse
%     \@safe@activestrue\edef\@tempe{#3}\@safe@activesfalse
%     \org@@citex{\@tempd}[#2]{\@tempe}}%
%   \makeatother
%\addbibresource{MyBiblio.bib}

%\makeatletter
% \def\url@leostyle{%
%   \@ifundefined{selectfont}{\def\UrlFont{\sf}}{\def\UrlFont{\footnotesize\ttfamily}}}
% \makeatother
%% Now actually use the newly defined style.
% \urlstyle{leo}

\usepackage{relsize}
\appto{\bibsetup}{\emergencystretch=1em}
\renewcommand*{\UrlFont}{\itshape}
\appto{\biburlsetup}{%
  \renewcommand*{\UrlFont}{\smaller[0.5]
    \itshape% \ttfamily
    %\normalfont\itshape
  }}

\makeindex

\usepackage{titling}
\title{Compte-rendu de réunion}
\author{Luc Sapin}
\date{\today}

%Package pour avoir des liens hyper-texte dans les fichiers Pdf 
%A mettre en dernier 
%Pose parfois problème 
% \usepackage[unicode,final% ,ocgcolorlinks
% ]{hyperref}
% \usepackage{footnotebackref}

\usepackage[final]{hyperref}

\hypersetup{%
  pdfinfo={
    Title={\thetitle},
    Author={\theauthor} },
  % frenchlinks,
  linkcolor=purple,
  citecolor=teal,
  %ocgcolorlinks
  %urlcolor={magenta!50!black},
  % menucolor=purple
}
\ifprinted
\hypersetup{colorlinks=false}
\else
\hypersetup{colorlinks}
\fi



% \tikzexternalize[mode=list and make,up to date check=md5, prefix=figures/]
% \pgfkeys{/pgf/images/include external/.code={\includegraphics{#1}}}


\begin{document}
\maketitle
\chapter*{}
\label{cha:cr-reu}
\section*{Vendredi 23 mars 2018}
Correction des objectifs \& de leur priorité :
\begin{itemize}
	\item Changement $EMB \rightarrow L_2$ pour l'aller, le retour et l'aller-retour. Lancer optimisation du modèle impulsionnel sur quelques astéroïdes en temps min seulement;
	\item Construire algorithme de classification sur la base de données des 4000+ astéroïdes;
	\item Étude de la phase de parking dans la sphère de Hill : modifier dynamique et résoudre problème de conso min. 
\end{itemize} 

\section*{Vendredi 6 avril 2018}
\subsection*{Priorité semaine prochaine}
\begin{itemize}
	\item rendre aller \& aller-retour fonctionnels pour le point L2 (au départ et à l'arrivée avant de comparer les dynamiques dans la sphère de hill.
	\item Regarder BOCOP avec l'interface graphique, comprendre les commandes matlab qui appellent BOCOP pour la phase parking (user guide sur \href{bocop.org}{bocop.org}) avant d’entamer la phase parking en temps min.
\end{itemize}

\subsection*{Priorité plus long terme}
\begin{itemize}
	\item comparaison des solutions entre les points L2 et EMB : tableau comparatif sur une dizaine d'astéroïdes
	\item comparaison des solutions avec critères initial et critère sans le "max".
	\item Étude de la phase de parking dans la sphère de Hill : modifier dynamique et résoudre problème de conso min en contrôle optimal.
	Idée : impulsionnel sur la phase parking : comparer avec la solution en contrôle optimal. Si solutions proches, voir optimisation globale en impulsionnel ?
	\item Construire algorithme de classification sur la base de données des 4000+ astéroïdes; 
\end{itemize}

\newpage
\section*{Vendredi 13 avril 2018}
\subsection*{Priorité semaine prochaine}
\begin{itemize}
	\item Affichage trajectoires avec la dynamique 3 corps modifiés
	\item comparer pour une dynamique donnée les résultats entre EMB et L2
	\item tableau comparatif sur une dizaine d'astéroïdes : les résultats de l'impulsionnelle (deltaV, date arrivée, date départ,...)  et les résultats de Bocop  sur temps min (solution, données de convergence,...)
\end{itemize}

\section*{Vendredi 20 avril 2018}
\begin{itemize}
	\item Essayer de prolonger l'intégration jusqu'au bout ($\rightarrow dist = 0$) \& comparer résultats avec $dist = 0.01$;
	\item convertir positions \& trajectoires dans repère tournant;
	\item regarder incohérence temps départs < temps arrivé;
	\item faire pareil avec les résultats de Bocop en temps min : trajectoires \& résultats (graphes comparatifs);
	\item enlever le max et comparer $\Delta V$ : L2 avec / sans le max;
	\item impulsionnelle sur phase parking : poser le problème;
	\item optimiser tout ensemble si ça marche bien;
	\item le faire sur tout les astéroïdes.
\end{itemize}

\section*{Jeudi 26 avril 2018}
\begin{itemize}
	\item Propager la dynamique 2 corps jusqu'au bout et comparer résultats
	\item observer les trajectoires des résultats de BOCOP
	\item poser le problème : impulsionnel sur la phase parking vers le point L2. Attention : il faut intégrer pour avoir la position du spacecraft.
	\item optimiser tout ensemble si ça marche bien
	\item le faire sur tous les astéroïdes
\end{itemize}

\end{document}

%%% Local Variables:
%%% mode: latex
%%% TeX-master: t
%%% End: